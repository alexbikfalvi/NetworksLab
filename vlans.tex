\chapter{Virtual LANs (VLANs)}

\section{Switch user manual}

The user manual for the switch is available here:

\url{http://www.jaumebarcelo.info/teaching/lxs/wlan/manual_vlan.pdf}


\section{Introduction}

In this lab assignment the switches will be configured to create different VLANs.
The IPs of the switches are 192.168.1.110, 192.168.1.111 and 192.168.1.112.
An sketch on the blackboard specifies the switch to which your PC is connected.
Each VLAN has a unique identifier that takes values between 0 and 4094.
In this assignment we will use the identifiers 10 and 20.


Each group will use three PCs.
One of the PCs is for managing the switch and needs an IP address of the same range as the IP of the switch (check the blackboard for the details).
The IPs of the other two switches must belong to the range of the VLAN that you are going to use (192.168.\{10,20\}.XX).

%% Command to view the serial COM ports: "compmgmt.msc"

\section{Creation of a VLAN}

\begin{table}[!t]
%% increase table row spacing, adjust to taste
\renewcommand{\arraystretch}{1.3}
% if using array.sty, it might be a good idea to tweak the value of
%\extrarowheight as needed to properly center the text within the cells
\caption{Command modes}
\label{tab:modes}
\centering
\begin{tabular}{|c|p{5cm}|c|}
\hline
Command Mode & Access & Prompt\\
\hline
User EXEC & Connecting to the switch & Switch> \\
Privileged EXEC & Using the enable command in the ``User EXEC'' mode & Switch\# \\
Global Configuration & Using the ``configure terminal'' command in the ``Privileged EXEC'' mode& Switch(config)\# \\
        Interface Configuration & Using the ``interface $<$interface-name$>$'' command in the ``Global Configuration'' mode specifying the interface that we want to configure, e.g. FastEthernet0/4& Switch(config-if)\# \\
\hline
\end{tabular}
\end{table}


Table \ref{tab:modes} describes the four possible modes of interaction with the switch.
The first one offers limited information about the switch.
It is possible to find more detailed information in the second (privileged EXEC) mode.
The third mode allows the configuration of general aspects of the switch while and the last one is to configure a specific interface.
The commands available in each of the modes are different.
Make sure you are in the right mode before issuing a command.
It is possible to move to the previous mode using the ``exit'' command.

Use a telnet client to connect to the switch and observe which is the initial mode.
You can use the command ``?'' to obtain information about the possible commands in a given mode.
Additionally, you can also follow a partial command by ``?'' to obtain more information about how to use the command and the required parameters.
For example ``ip address ?'' would give you information about the parameters you could use after address.

Enter the mode \emph{privileged} EXEC and use the command \texttt{show running-config} to see the current configuration of the switch.
Answer the following questions.
\emph{How many VLANs can you observe? (Note that this is not necessarily the number of VLANs in the switch).}
\emph{How many Fast Ethernet interfaces are available?}
\emph{What is the VLAN1 administrative address?}

There exists a \emph{default} VLAN which has the number 1.
Use the command \texttt{show vlan} or \texttt{show vlan id \textless id \textgreater}.
\emph{What is the status of VLAN1?}
\emph{How many VLANs are there in the switch?}
For each of the VLANs identify the ID, the name, the status, the assigned ports and the type.
Include this information in the report.

Enter the config mode and try to delete the default VLAN:
\texttt{Switch-B(config)\#no vlan 1}.
\emph{What happens?}

In the config mode, use the ``?'' to find which commands can be used in this mode.
Use the \texttt{vlan <id>} to create a new VLAN.
The type is Ethernet and the id must be set according to the sketch on the blackboard.
Include the exact command that you used and the reply message of the router in your report.

Verify the new configuration using the command \texttt{show vlan} in the privileged EXEC mode.
\emph{What is the default name of the new VLAN?}

Delete the VLAN that you have just created using \texttt{no vlan <id>}, verify that it has been deleted and created again.
Include in your report the sequence of commands that you used and the output of the switch after each command.
You can use the command \texttt{show vlan brief}.

In the global configuration mode, use the command \texttt{vlan <id>} to configure the VLAN that you have created.
In the VLAN configuration mode use the name command to change the name of the VLAN \texttt{Switch-B(config-vlan)\#name test}.

Name your VLAN vlanXX-GroupX-switchX.
Verify the changes.
Include in your report the exact commands that you used and the output of the switch.
\emph{Which other parameters can be changed in the vlan configuration mode?}

In the privileged EXEC mode, take a look at the running configuration and compare it with the start-up configuration.
\emph{Are they equal?}
Find which is the command that is needed to copy the running configuration to the start-up configuration.
You will need this command when you make changes to the configuration that you want to save.

\section{Static assignment of ports to a VLAN}
\label{sec:static}

After having created one or more VLANs, the next step is to assign ports to the VLANs.
The simplest assignment is the ``static'' assignment.

Go to the global configuration mode.
Find out which are the ports that you want to modify (e.g., 0/1, 0/2, etc.).
Change the ports to which your two auxiliary PCs are connected.
Make sure that you do not change the port that you are using for the telnet connection.

To make the changes, you first have to go into to interface configuration mode.
Now check the options of the \texttt{switchport} command and use the switch user manual if you require extra information.

After making the changes go back to the privileged execution mode.
Verify which are the changes that you have made comparing the running config to the start-up config.
Save your changes.

Use the \texttt{ping} command to test the connectivity between the two auxiliary PCs.
Then try the connectivity between the configuration PC (the one with the telnet connection) to the auxiliary PCs.
Finally, try the connectivity between your PCs and other groups PCs.
Explain the results and the conclusions of the experiments.

Try to complete Table \ref{tab:connectivity}.

\begin{table}[!t]
%% increase table row spacing, adjust to taste
\renewcommand{\arraystretch}{1.3}
% if using array.sty, it might be a good idea to tweak the value of
%\extrarowheight as needed to properly center the text within the cells
\caption{Command modes}
\label{tab:connectivity}
\centering
\begin{tabular}{|c|c|c|}
\hline
& Same Switch & Different Switch\\
\hline
Same VLAN & ok/ko & ok/ko \\
\hline
Different VLAN & ok/ko & ok/ko \\
\hline
\end{tabular}
\end{table}

\emph{Which devices are reached if you use a broadcast packet?}
\emph{How can a packet travel from one VLAN to a different VLAN?}

\section{Trunk ports}

A trunk port can carry traffic of different VLANs between two switches.
You will find which are the trunking ports on the blackboard.
In the privileged EXEC mode use the command \texttt{show interfaces <trunk interface> switchport}.
Write down the following parameters:
\begin{itemize}
\item Administrative mode
\item Operational mode
\item Administrative trunking encapsulation
\item Trunking native mode
\item Trunking VLAN enabled
\item Trunking VLAN active
\end{itemize}

Use the \texttt{show vlan} command to check the status of the ports.
\emph{Where can you find the trunk ports?}

\section{Defining the VLANs carried by a given trunk port}

By default, a trunk port carries traffic of all VLANs.
It is possible to configure which VLANs are allowed in a given trunk port.

From the privileged EXEC mode, check which are the VLANs in the trunk port of your switch.
\emph{Which command do you use?}

In the global configuration mode, enter into the configuration of the trunk port.
Specify the command that you use.

Now we are going to configure the trunk port to allow the traffic of our VLAN.
Check the options of the command \texttt{switchport trunk allowed vlan [remove,add] vlan-list}.
The parameter \texttt{vlan-list} is a list of VLAN identifiers (or names) separated by a  hyphen (or a comma, respectively).

Exit the configuration mode and move back to the privileged EXEC mode.
Use the commands \texttt{show interface interface-id status} or \texttt{show interface status} to see the configuration of one or all the interfaces.
\emph{What are the results?}.

Try also the \texttt{show interfaces trunk} command and write down the results.

\section{Connectivity test}

Verify whether the following connections are possible and explain why.
\begin{itemize}
\item Ping a PC of the same VLAN connected to a different switch.
\item Ping a  switch of a different VLAN.
\item Perform additional tests.
\end{itemize}

Compare the results that you obtain now with the results obtained in Section \ref{sec:static}.
Fill in the connectivity table again.

\emph{Are there any differences? Why}

Change the IP address of one of your auxiliary PCs to an address belonging to the range of the other VLAN.
Perform connectivity tests again.
\emph{What happens? Why?}

\section{Network topology}

Draw the network topology, both from the physical point of view and the logical point of view.

\section{Preparing the report}

These are aspects that you may want to cover in your report:
\begin{itemize}
\item What is a VLAN and what is used for?
\item What are the differences among the different modes of the switch?
\item Relation between the active VLANs and the different ports.
\item Differences between access ports and trunk ports.
\item Connectivity in the different situations.
\item Remote management using telnet and VLAN 1.
\end{itemize}


\section{Changing the native VLAN (Optional)}

For security reasons, it is recommended to change the native VLAN (e.g., to 666) of the switches and leave no ports assigned to the VLAN, except for administration.

\section{Speed and duplexing (Optional)}

Change the speed of the port and the duplexing and perform tests using iperf.

\section{Administrative shutdown of interface (Optional)}

Try to administratively disable access ports and trunking ports and describe the results.










