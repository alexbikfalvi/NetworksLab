\chapter{Traffic Analysis}

\section{Introduction}
The goal of this lab assignment is to know and use monitoring and traffic analysis tools.
\emph{Wireshark} and \emph{tcpdump} will be used to study different layers of the TCP/IP architecture.

\section{Preparation at home}

Review the TCP/IP model and explain the function of each layer.
Provide examples of the protocols at each layer of the protocol stack.

What is the purpose of ARP?
Draw a sketch of the different messages being exchanged and the different steps involved.
Is it possible to run this protocol between computers that are in different local area networks (LANs)?

What is the ICMP protocol?
How does the \emph{ping} command work?
What does the \emph{ping} command measure?

Explain and draw an SSL connection indicating how the protocol works and which messages are being exchanged.

\section{WireShark Network Analyzer}

Start the WireShark software and choose the right network interface.
The option Capture/Interfaces starts the packet capture.
It is also possible to configure the length of the capture and other details.
What interface does WireShark detect?
What is your IP address?
And the corresponding MAC?

Configure the Capture/Interfaces options to perform a five minutes capture.
Observe the results and answer the following questions:

What is the total number of captured packets?
Are there lost packets?
If yes, why?

Choose any packet.
Observe the details and answer the following questions:
What is the source and destination IP address?
What are the source and destination MAC addresses?
What is the number of bytes in the packet?
What protocols can you see in the packet?
Is there HTTP?
If yes, what is the length of the HTTP message?
What are the source and destination port?

In the menu ``Analyze option/Enable protocol'' it is possible to configure the protocols that WireShark will capture and show.
Looking at the default protocols, find at least one protocol of each of the four upper layers of the TCP/IP stack (Application/Transport/Internet/Link).
Include a brief description of the protocols you found.

Go to the menu ``Statistics/Protocol Hierarchy'' and observe the percentage of the following protocols: Ethernet, Internet Protocol, TCP, UDP, Logical Link Control, ARP, STP, IPv6, HTTP.

What are the differences between IP and IPv6?

\section{the ARP protocol}
The Address Resolution Protocol (ARP) protocol resolves the association between an IP address and a MAC address.
It is used in IP over Ethernet networks.
Capture traffic and analyze the ARP packets.
You can filter the ARP packets writing ``ARP'' in the ``Filter Toolbar''.

What are the source and destination MAC addresses of the Ethernet frame that contains the ARP request message?
Can you see the source and destination IP addresses in the ARP request frame?

Look for an ARP request-reply exchange and write the source and destination MAC and IP addresses.

Whats the time elapsing between an ARP request and reply messages?

Use the information available in WireShark to indicate the length of the ARP frames and draw the format of the messages.

To which layer does ARP belong?

\section{HTTP and secure HTTP}

Make a new 5 minutes capture and during this time visit a few webpages.
After the capture is finished observe the different HTTP and HTTPS messages.
Use the filter toolbar to filter the messages.

Observe an HTTP GET message and the corresponding response and answer the following questions:

What is the HTTP version of your web browser?
And the HTTP version of the server?
What language does the client request to the server?

Is it possible to find which are the URLs visited by the user?
At which layer is this information available?

The default destination port for web is 80.
What is the source port of the get requests?
Write the source number for different connections.
At which layer can you find this information?

Find a DNS query/response pair.
What is the function of DNS?

Use the option ``Analyze -> Follow TCP Stream'' to analyze a TCP session.
Identify the three-way-handshake and the session tear-down.

If http is used, it is possible to observe the contents of the web using WireShark?

Now use HTTPs.
Is it still possible to read the information that is being transmitted?
(Look for SSL packets).

Identify a SSL handshake in WireShark.

\section{ICMP-ping packet capture (Homework)}

Close all the applications that use the network and ping four different webs in four different continents.
Analyze the results.

What protocols are being use?
Draw a frame and explain how the different packet are encapsulated in each other.

How many ping messages are transmitted by default?

Prepare a table with the source, destination, and average packet delay  of the four different ping experiments.

What packet length is being used?

At which layers can we find source and destination addresses?
Which kind of addresses?

Are the ping packets sent uniformly in time?
What about the answers?
What are the reasons for different inter-arrival times for the answers?

What information is included in the data field of the ICMP packets?
What about in the reply messages?

\section{tcpdump}

In this section we will use the tcpdump command in linux.
Use ``man tcpdump'' to learn about the different parameters and options.
With tcpdump it is also possible to filter the traffic according to the source or destination addresses, protocol, port number, etc.

Open a terminal and launch a tcpdump capture.
Finish the capture using ``Ctrl-C''.
What is the information provided by tcpdump and which format is being used?
To which level does the information belong?
(Remember that you can redirect the output using \$ tcmpdump $>$ my-file ).

The first line of tcpdump specifies which interface is being used and it can be changed using the -i option.
What interface are you using?

Describe the information provided for the ARP protocol (tcpdump arp).

Execute the same command again using the -e flag.
Whats the difference with respect to the previous execution?
Check the tcpdump manual if necessary.

Try new captures related to this assignment, such as ``tcpdump stp'', ``tcpdump http'', ``tcpdump http'', ``tcpdump udp'', ``tcpdump ssl'', ``tcpdump ip'', etc.
Try also to make captures for a specific IP address.
