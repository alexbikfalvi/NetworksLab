\chapter{Traffic Analysis, LAN, WLAN}

\section{Layered Networks}
Data networks are organized in layers.
These layers communicate with each other using interfaces.
Each layer offers some services to the layer on top of it.
Also, each layer in one end of the communication connects to the same layer at the other end of the communication.

In theory, the layers are relatively independent from each other.
A layer can see other layers as \emph{black boxes} without having to care about implementation details.
This approach is a common practice in engineering as it allows each team of engineers to focus in a particular problem which is just a piece of a puzzle to solve a bigger problem.

Each layer accomplishes some particular tasks.
For example, the physical layer transmits symbols containing information over a transmission medium such as an optical fiber or a radio channel.
The network layer is assigned the task of taking routing decisions to move packets from one network to the other.

The Open Systems Interconnection model (OSI) partitions a communication system in seven layers: physical, data link, network, transport, session, presentation and application.
The TCP/IP model (Transfer Control Protocol/Internet Protocol) is more specific to the Internet and differentiates five layers: physical, link, network, transport, and application.`

The physical layer deals with the transmission medium and transfers information symbols over it.
The link layer is in charge of one-hop communication.
The network layer is responsible for connecting multiple networks and therefore should be capable of routing data through multiple hops.
The transport network takes care of end-to-end communication between two hosts, and can multiplex different instances of communications in each host.
The application layer uses the end-to-end communication to offer some kind of service to the network user.
